\documentclass[10pt,pdf,hyperref={unicode},xcolor=dvipsnames]{beamer}

\usepackage{float}
\usepackage{amsmath}
\usepackage{amsfonts}
\usepackage{bbm}

% русский текст в формулах?
\usepackage{mathtext}

% русский
\usepackage[T2A]{fontenc}
\usepackage[english]{babel}
\usepackage[utf8]{inputenc}

% рисунки
\usepackage{graphicx, caption, subcaption}

\usetheme{CambridgeUS}

\usepackage{csquotes}
\usepackage[backend=bibtex]{biblatex}
\bibliography{biblio}

\usepackage{physics}
\usepackage{braket}

% ?
\setbeamertemplate{frametitle}[default][center]

\addtobeamertemplate{navigation symbols}{}{%
\usebeamerfont{footline}%
\usebeamercolor[fg]{footline}%
\hspace{1em}%
\large\insertframenumber/\inserttotalframenumber
}

\newcommand{\mytitle}[1]{\color{blue}{\textbf{#1}}}

\newcommand{\lb}{\left(}
\newcommand{\rb}{\right)}
\newcommand{\lsq}{\left[}
\newcommand{\rsq}{\right]}

\usepackage{dsfont}
\usepackage{bbm}

\newcommand{\bbA}{\mathbb{A}}
\newcommand{\bbB}{\mathbb{B}}
\newcommand{\bbV}{\mathbb{V}}
\newcommand{\bbH}{\mathbb{H}}

\newcommand{\lc}{\left\{}
\newcommand{\rc}{\right\}}

\newcommand{\psip}[1]{\psi^{(#1)}(x)}
\newcommand{\fx}[0]{f(x)}
\newcommand{\psix}[0]{\psi(x)}

% slightly modifying gather environment
\let\oldgather\gather
\def\gather{\vspace*{-0.2cm}\oldgather}

\begin{document}

\begin{frame}{\center\mytitle{\Large Расчет классических и квантовых статистических сумм слабосвязанных систем}}
\begin{table}[]
\flushright
\begin{tabular}{r}
\large Финенко Артем \\[1ex]
\end{tabular}
\end{table}
\vfill
\center
\today
\end{frame}
   
\begin{frame}{Структура доклада}
    \begin{block}{}
        \begin{itemize}
            \item Два основных подхода к одномерной задаче Штурма-Лиувилля \\
            \item Метод Нумерова и его матричный аналог \\
            \item Обобщенный матричный метод Нумерова \\
            \item Метод Нумерова и трехточечная формула. Асимптотические разложения для собственных значений \\
            \item Экстраполяция Ричардсона для собственных значений \\
            \item Расчет колебательно-вращательных уровней в потенциале Морзе \\
            \item Классическая статистическая сумма \\
            \item Сравнение статсумм 
        \end{itemize}
    \end{block}
\end{frame}

\begin{frame}{Метод Нумерова и его матричный аналог}
    \begin{block}{}
        Метод Нумерова -- численный метод, позволяющий решать дифференциальные уравнения второго порядка\footnote{Метод Нумерова допускает ненулевой свободный член в ДУ}
        \begin{gather}
            \psip{2} = f(x) \psi(x), \quad f(x) = -\frac{2 m}{\hbar^2}\lsq E - V(x) \rsq, \quad \psip{n} = \frac{d^n}{dx^n} \psi(x).
        \end{gather}
        Используя Тейлоровское разложение для волновой функции
        \begin{gather}
            \hspace*{-0.1cm}
            \psi(x \pm h) = \psi(x) \pm h \psip{1} + \frac{1}{2!} h^2 \psip{2} \pm \frac{1}{3!} h^3 \psip{3} + \frac{1}{4!} h^4 \psip{4} + O(h^5), \notag
        \end{gather}
        получим выражение для второй производной $\psip{2}$ с точностью до $O(h^4)$
        \begin{gather}
            \psip{2} = \frac{ \psi(x+h) + \psi(x-h) - 2\psi(x) }{ h^2 } - \frac{1}{12} h^2 \psip{4} + O ( h^4 ).
        \end{gather}
    \end{block}
\end{frame}

\begin{frame}{Метод Нумерова и его матричный аналог}
    \begin{block}{}
        Используем это выражение для получения четвертой производной $\psip{4}$ с точностью до $O(h^2)$
        \begin{gather}
            \psip{4} = \frac{d^2}{dx^2} \psip{2} = \frac{d^2}{dx^2} \lsq \fx \psix \rsq = \notag \\
                     = \frac{ f(x+h)\psi(x+h) + f(x-h)\psi(x-h) - 2f(x)\psi(x) }{ h^2 } + O(h^2). 
        \end{gather}
        Подставляем в выражение для второй производной (суммарный порядок остается $O(h^4)$)
        \begin{gather}
            \fx \psix = \frac{\psi(x+h)+\psi(x-h)-2\psi(x)}{h^2} - \hspace{5cm} \notag \\ 
            - \frac{f(x+h) \psi(x+h) + f(x-h) \psi(x-h) - 2f(x) \psi(x)}{12} + O(h^4).
        \end{gather}
    \end{block}
\end{frame}

\begin{frame}{Метод Нумерова и его матричный аналог}
    \begin{block}{}
        При пропагировании на сетке используют вспомогательную функцию 
        \begin{gather}
            \omega(x) =  \lb 1 - \frac{h^2}{12} \rb \psi(x) \\
            \omega(x + h) = 2\omega(x) - \omega(x-h) + h^2 f(x) \psi(x).
        \end{gather}
        Вводя обозначения
        \begin{gather}
            V_{i-1} \equiv V(x-h), \quad V_i \equiv V(x), \quad V_{i+1} \equiv V(x + h) \\
            \psi_{i - 1} \equiv \psi(x-h), \quad \psi_i \equiv \psi(x), \quad \psi_{i+1} \equiv \psi(x+h),
        \end{gather}
        получаем следующее выражение, удобное для матричной техники 
        \begin{gather}
            \hspace*{-0.3cm}
            -\frac{\hbar^2}{2 m} \frac{\psi_{i+1} + \psi_{i-1} - 2 \psi_i}{h^2} + \frac{V_{i+1} \psi_{i+1} + V_{i-1}\psi_{i-1}+10 V_i \psi_i}{12} = E \frac{\psi_{i-1} + 10 \psi_i + \psi_{i+1}}{12}. \notag 
        \end{gather}
    \end{block}
\end{frame}

\begin{frame}{Метод Нумерова и его матричный аналог}
    \begin{block}{}
        Матричная формулировка метода Нумерова 
        \begin{gather}
            -\frac{\hbar^2}{2 m} \bbA \psi + \bbB \bbV \psi = E \bbB \psi, \\
            \psi = \lsq \psi_i, i = 1 \dots N \rsq^\top, \quad \bbV = \text{diag} \left\{ V_i, i = 1 \dots N \right\} \\
            \bbA = \frac{1}{h^2} 
            \begin{bmatrix}
                -2 & 1 & 0 & 0 & \dots \\
                 1 &-2 & 1 & 0 & \dots \\ 
                 0 & 1 & -2& 1 & \dots \\
                 0 & 0 & 1 & -2& \dots \\
                 \vdots & \vdots & \vdots & \vdots & \ddots
            \end{bmatrix}, \quad
            \bbB = \frac{1}{12} 
            \begin{bmatrix}
                10 & 1 & 0 & 0 & \dots \\
                1 & 10 & 1 & 0 & \dots \\
                0 & 1 & 10 & 1 & \dots \\
                0 & 0 & 1 & 10 & \dots \\
                \vdots & \vdots & \vdots & \vdots & \ddots
            \end{bmatrix} \\
            \bbH \psi = E \psi, \quad \bbH = -\frac{\hbar^2}{2m} \bbB^{-1} \bbA + \bbV
        \end{gather}
    \end{block}
\end{frame}

\begin{frame}{Обобщенный метод Нумерова}
    \begin{block}{}
        \vspace*{-0.5cm}
        Для получения метода порядка $N = 2r + 2$, выразим вторую производную $\psip{2}$ до порядка $N$ 
        \begin{gather}
            \psi(x+h) + \psi(x-h) = 2 \psi(x) + \sum_{k=1}^{r+1} \frac{ 2h^{2k} }{ (2k)! } h^{2k} + O(h^{2r+4}), \\
            \psip{2} = \frac{ \psi(x+h) + \psi(x-h) - 2\psi(x) }{ h^2 }  - \sum_{k=0}^{r-1} \frac{ 2h^{2k+2} }{ (2k+4)! } \psip{2k+4} + O(h^{2r+2}). \notag
        \end{gather}
        Неизвестными являются производные $\lc \psip{2k + 4}, k = 0 \dots r-1 \rc$, которые мы найдем из системы линейных уравнений
        \begin{gather}
            \lc
            \begin{aligned}
                \frac{ \psi(x+h) + \psi(x-h) - 2\psi(x) }{ 2 } &= \sum_{k=1}^{r} \frac{ h^{2k} }{ (2k)! } \psip{2k} + O(h^{2r+2}) \\ 
                                                               &\dots \\
                \frac{ \psi(x+r\cdot h) + \psi(x-r \cdot h) - 2\psi(x) }{ 2 } &= \sum_{k=1}^{r} \frac{ (r \cdot h)^{2k} }{ (2k)! } \psip{2k} + O(h^{2r+2}) \\ 
            \end{aligned}
            \right.
        \end{gather}
    \end{block}
\end{frame}

\begin{frame}{Обобщенный метод Нумерова}
    \begin{block}{}
        \vspace*{-0.5cm}
        В результате решения линейной системы получаем наборы коэффициентов $\lc c_i \rc_{i=1}^r$, $\lc k_i \rc_{i=1}^r$, позволяющие получить выражения
        \begin{gather}
            \psip{2} = \frac{1}{h^2} \sum_{i=-r}^r c_i \psi_i - \frac{2 h^{2r}}{(2r+2)!} \psip{2r+2} + O(h^{2r+2}) \\
            \psip{2r} = \frac{1}{h^{2r}} \sum_{i=-r}^r k_i \psi_i 
        \end{gather}
        Воспользуемся приемом из стандратного метода Нумерова для нахождения $\psip{2r+2}$
        \begin{gather}
            \psip{2r+2} = \frac{d^r}{dx^r} \lb f(x) \psi(x) \rb = \frac{1}{ h^{2r} } \sum_{i=-r}^r k_i f_i \psi_i.
        \end{gather}
        Собирая полученные выражения, получаем уравнения обобщенного метода Нумерова 
            \vspace*{-0.4cm}
        \begin{gather}
            \frac{1}{h^2} \sum_{i=-r}^r c_i \psi_i = f_i \psi_i + \sum_{i=-r}^r \frac{2}{(2r+2)!} k_i f_i \psi_i.
        \end{gather}
    \end{block}
\end{frame}

\begin{frame}{Обобщенный метод Нумерова}
    \begin{block}{Пример. Порядок $N = 8, (r = 3)$.}
        \vspace*{-0.5cm}
        \begin{gather}
            \bbH \psi = E \psi, \quad \bbH = -\frac{\hbar^2}{2m} \bbB^{-1} \bbA + \bbV \\
            \bbA = \frac{1}{180h^2}
            \begin{bmatrix}
                490 & 270 & -27 & 2 & 0 & \dots \\
                270 & 490 & 270 & -27 & 2 & \dots \\
                -27 & 270 & 490 & 270 & -27 & \dots \\
                  2 & -27 & 270 & 490 & 270 & \dots \\
                  0 & 2   & -27 & 270 & 490 & \dots \\
                  \vdots & \vdots & \vdots & \vdots & \vdots & \ddots
            \end{bmatrix}, \\
            \bbB = \frac{1}{20160}
            \begin{bmatrix}
                20140 & 15 & -6 & 1 & 0 & \dots \\
                15 & 20140 & 15 & -6 & 1 & \dots \\
                -6 & 15 & 20140 & 15 & -6 & \dots \\
                1 & -6 & 15 & 20140 & 15 & \dots \\
                0 & 1 & -6 & 15 & 20140 & \dots \\
                \vdots & \vdots & \vdots & \vdots & \vdots & \ddots
            \end{bmatrix}
        \end{gather}
    \end{block}
\end{frame}

\begin{frame}{Метод Нумерова и трехточечная формула}
    \begin{block}{}
        Использование трехточечной формулы для второй производной $\psip{2}$ приводит к матричной задаче, похожей на Нумеровскую \footfullcite{goorvitch1992}
        \begin{gather}
            -\frac{ \hbar^2 }{ 2m } \psip{2} + V(x) \psi(x) = E \psi(x), \\
            \psip{2} = \frac{\psi(x+h) + \psi(x-h) - 2\psi(x)}{h^2} + O(h^2) \\
            \bbH = - \frac{\hbar^2}{2m} \bbA + \bbV \\
            \bbA = \frac{1}{h^2} 
            \begin{bmatrix}
                -2 & 1 & 0 & 0 & \dots \\
                1 & -2 & 1 & 0 & \dots \\
                0 & 1 & -2 & 1 & \dots \\
                \vdots & \vdots & \vdots & \vdots & \ddots
            \end{bmatrix}
        \end{gather}
    \end{block}
\end{frame}

\end{document}
